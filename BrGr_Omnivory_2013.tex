\documentclass[12pt,a4paper,oneside]{article}
\renewcommand{\baselinestretch}{1.2}
\usepackage{sectsty,setspace,natbib,wasysym} 
\usepackage[top=1.00in, bottom=1.0in, left=1in, right=1.25in]{geometry} 
\usepackage{graphicx}
\usepackage{latexsym,amssymb,epsf} 
\usepackage{epstopdf}
\usepackage{exceltex}
\usepackage{lineno}
\usepackage{rotating}
\usepackage{amsmath}
\usepackage{natbib}


\parskip=5pt
\pagenumbering{arabic}
\pagestyle{plain}
 
\newcommand{\Section}[1]{\vspace{-8pt}\section{\hskip -1em.~~#1}\vspace{-3pt}} 
%\usepackage{ametsoc}

\begin{document}
{ \noindent \bf Linking the green and brown worlds: The prevalence and
 effect of multi-channel feeding in food webs}\\
\\
\noindent {\bf Authors:} \\
\noindent Elizabeth M. Wolkovich\(^{1,2,3*}\), Stefano Allesina\(^{4}\), Kathryn
L. Cottingham\(^{3}\), John C. Moore\(^{5,6}\), Stuart A. Sandin\(^{7}\) and Claire de
Mazancourt\(^{8}\)\\
\\
\noindent \(^{1}\) University of British Columbia, 6270 University Blvd., Vancouver, BC, V6K 1TZ, Canada (\emph{lizzie@biodiversity.ubc.ca})\\
\noindent \(^{2}\) University of California, San Diego, 9500 Gillman
Drive \#0116, La Jolla, CA 92093 USA\\
\noindent \(^{3}\) Dartmouth College, Department of Biological
Sciences, 78 North College St., Hanover NH 03755,
USA (\emph{kathryn.cottingham@dartmouth.edu})\\
\noindent \(^{4}\) University of Chicago, Department of Ecology \&
Evolution, Computation Institute, Chicago, IL 60637 USA (\emph{sallesina@uchicago.edu})\\ 
\noindent \(^{5}\) Colorado State University, National Resource Ecological Laboratory, Fort
Collins, CO 80523 USA \newline
(\emph{jcmoore@nrel.colostate.edu})\\ 
\noindent \(^{6}\) Colorado State University, Department of Ecosystem Science and Sustainability, Fort Collins, CO 80523\\
\noindent \(^{7}\) Scripps Institution of Oceanography, University of
California - San Diego, 9500 Gilman Drive, La Jolla, CA 92093 USA
(\emph{ssandin@ucsd.edu}) \\
\noindent \(^{8}\) Centre for Biodiversity Theory and Modelling, Station d'Ecologie Exp\'{e}rimentale du CNRS, 09200 Moulis, France 
(\emph{claire.demazancourt@ecoex-moulis.cnrs.fr})\\
\noindent * Corresponding author
\\

\noindent Prepared as an {\bf article} for \emph{American
 Naturalist}\\

\noindent {\bf Keywords:} food webs, stability, detritus, green world, attack
rates, multi-channel\\

\noindent {\bf Supplements for online only:}\\
\noindent \emph{Appendix A:} Complete list of empirical food webs used.\\
\noindent \emph{Appendix B:} Effect of varying parameters used to
estimate trophic level and diet. 
\noindent \emph{Appendix C:} Expanded version of Table 1, including references. \\

 

\pagebreak
\linenumbers
\modulolinenumbers[2]

\noindent {\bf Abstract}\\
\noindent Recent advances in food web ecology highlight that most real food webs (1) represent an interplay between producer and detritus-based webs and, (2) are governed by consumers which are rampant omnivores. A possible avenue to unify these advances comes from models demonstrating that predators feeding on distinctly
different channels may stabilize food webs. Empirical studies suggest many consumers engage in such behavior by feeding on prey items from both living-autotroph (green) and
detritus-based (brown) webs---what we term `multi-channel feeding'---but we know little about how common such feeding is across systems and trophic levels and its effect on system stability. Considering 23 empirical webs, we find that multi-channel feeding
is equally common across terrestrial, freshwater and marine systems and occurs most often
among primary consumers.  We next
developed a simple four-compartment nutrient cycling model for consumers
eating autotrophs and detritus, which showed that---across terrestrial and aquatic ecosystems---multi-channel feeding is
stabilizing at low attack rates on autotrophs, but destabilizing at high attack rates on autotrophs, compared to herbivory- or detritivory-only models. Together our results demonstrate that multi-channel feeding is common across ecosystems and may be a
stabilizing force in real webs that have consumers with asymmetric attack rates.

% Bring back if have more that 200 words someday:
% [INSERT after -only models.] However, the set of conditions for stable webs with multi-channel consumers is narrower for aquatic systems.
% [INSERT after rampant omnivores. ] ---feeding on varied prey across trophic levels and resource channels


\pagebreak
\noindent {\bf Introduction}\\
\\
\indent Reconciling the complexity of food webs with their apparent stability over time, and in response to disturbance, has driven a large body of research in theoretical community ecology \citep{Pimm:1982,Polis:1998}. Much of this work has used models of (living autotroph-based) grazing chains and webs \citep[the `green world,'][]{Hairston:1960kx} that include strong top-down control \citep{Pimm:1982,McCann:1997}. Such models have found many factors common in natural systems can destabilize systems \citep{Holt:1994,Tanabe:2005jc}. For example, feeding on multiple prey items (omnivory) in its many forms can be highly destabilizing \citep{Holt:1994}. In contrast, work focusing on detrital or `brown world' chains, often using aspects of donor-control, shows comparatively consistent stability \citep{Moore:2004,Blanchard:2011}. Models linking green and brown webs have traditionally done so by incorporating nutrient cycling, where dead materials from the green chain transfer to a detrital pool, which  mineralize into nutrients that limit the basal autotrophs of the green chain \citep{DeAngelis:1992, DeAngelis:1989}. Empirical food web studies, however, underscore that nutrient cycling is not the only connection between grazing and detrital webs.
\\
\indent Multi-channel \citep{Odum:1969,Moore:1988,Polis:1996} or multi-chain \citep{Vadeboncoeur:2005} feeding, where consumers link distinct resource channels, has been noted for some time \citep{Odum:1984}. Many omnivores are actually
multi-channel consumers that link grazing and
detrital channels, including species such as scorpions
\citep{Polis:1987}, predaceous nematodes and mesostigmatid mites \citep{Moore:1988}, wolf spiders \citep{Fagan:1997}, gizzard shad
\citep{Nowlin:2008}, and rocky littoral fish species
\citep{Pinnegar:2000}. In freshwater systems multi-channel consumers
that link mainly autotroph-based pelagic webs with highly
detritus-based benthic webs appear common and may drive trophic cascades
\citep{Vadeboncoeur:2002,Vadeboncoeur:2005}. Similar links between grazing and detrital-based chains
have been found in terrestrial soils \citep{Hunt:1987nw}, crop systems \citep{Settle:1996} and
forests \citep{Miyashita:2003}.
\\
\indent Given its prevalence in many systems, understanding how multi-channel feeding affects food web stability, especially in comparison to the more commonly-used models which have only grazing or detrital chains, may aid in explaining how complex natural food webs return to similar conditions following disturbance. Multi-channel consumers that link grazing and detrital resource channels may be especially
important because they provide a unique way for detrital
biomass to re-enter and affect the grazing web \citep{Polis:1996}. Such cross-chain feeding allows consumers to access detritus
directly (by eating detritus) and/or indirectly (by eating lower consumers
that are themselves detritivores or multi-channel consumers). This key link between
predation and the resource pool that is critical to nutrient cycling may
complicate the effect of multi-channel feeding on food web
stability: while detritus is often stabilizing when included in simple
food web models \citep{Moore:2004}, its role may change when predator
dynamics with top-down control are linked to more than one resource \citep{Holt:1994}.
\\
\indent Here we examine the prevalence of multi-channel feeding in real systems, and its role in the stability of modeled food webs. First, we examine whether multi-channel feeding is prevalent within and across real systems, using 23 empirical food
webs. Theoretical studies make mixed predictions regarding the
prevalence of multi-channel feeding, particularly whether it is more common at
higher versus lower trophic levels. Recent theory suggests that predators linking distinctly different energy
sources (i.e., fast and slow channels) should occur at higher trophic levels in food webs
\citep{Rooney:2006} and should derive their energy equally
across energy sources. However, earlier work suggested that distinct energy channels
based on living-autotroph versus detrital resources basically
break down after the first (basal) trophic level 
\citep{Odum:1984,Moore:1988ar,Moore:1988} as taxa consume both
living autotrophs and detritus. The sparse results to date are equivocal: in freshwater systems, the diets of
pelagic fish vary strongly by species, with some species deriving less
than 10\% of their diet from
alternative energy channels and others consuming considerable amounts
of benthic diet items \citep{Vander-Zanden:2002}. The empirical food web
data used here however allow tests of the commonness or rarity of
multi-channel feeding and its general trophic position in food webs across a wide array of ecosystem types. Next, we develop a simple food web model with nutrient
recycling to assess whether multi-channel feeding is a stabilizing or destabilizing component of food webs, in comparison with grazing-only or detrital-only models.
\\
\\
\noindent {\bf Prevalence of multi-channel feeding in real food webs}\\
\\
\emph{Methods}\\
\\
\indent We first examined data from real food webs to quantify the prevalence and
variation of multi-channel feeding across ecosystems and trophic levels.
Described food webs are inherently simplified versions of all actual
feeding relationships in a system and using such food webs to look for
actual ecological trends requires care \citep{Dunne:2004,
  Martinez:1991}. Here we attempted to control for possible bias by using the
best data available and choosing webs that are have well-resolved feeding relationships derived from robust sampling \citep{Martinez:1999}; this yielded
23 webs based on observation and gut content analysis. Total taxa per
web ranged from 21 to 200 and all webs included both detrital and
living autotroph taxa at the base. All webs gave links between
predators and prey (binary link data) while 13 also reported estimates
of the flows occurring between each resource and its
consumers. We classified webs as terrestrial, freshwater, emergent
vegetation (wetlands and mangroves)
or marine based on their taxa. A complete list of the webs used, and
their sources, is given in Appendix A.
\\
\indent We estimated the trophic level
and diet specificity for each consumer taxon in each web. We computed the trophic
level according to the flow
information \citep{Levine:1980} when present. For binary food webs,
we created a fractional diet matrix by assigning
to each of the \(X\) consumers of a given resource the same fraction
\(\varepsilon/X\) of the flow originating from the resource, where
\(\varepsilon\) is the efficiency of transformation. Using this
matrix, we then computed the trophic level as we did for the
flow-based food webs. For the figures reported here, we chose
\(\varepsilon=0.15\), but our analysis is not sensitive to the
particular value chosen for the efficiency of transformation (Appendix
B). For living autotrophs and detritus we
assigned a trophic level of 1. We then calculated the percent of diet
derived from detritus versus living autotrophs for each consumer on a scale
of 0 to 1, with 0 representing a diet derived solely of detrital-based
sources and 1 representing a completely green-web diet; this metric
included diet flows from feeding directly on living
autotrophs or detritus as well as diet flows from feeding on consumers
which themselves feed on both channels. We
operationally defined multi-channel consumers as taxa at or above the
second trophic level with diets falling between 0.1 to 0.9 on this
scale (results are not sensitive to this criterion, Appendix B). All
other taxa were then defined as detritus, living plants, or 
autotroph- or detritus-specialists depending on trophic level and
diet. We then determined the percent of taxa in each category for each
web, and tested for whether the prevalence of multi-channel consumers
varied by system type (terrestrial, freshwater, marine, emergent
vegetation) via a one-way ANOVA.
\\
\indent We then tested the prediction that consumers derive their food resources more
equally from brown and green channels
(i.e., are more omnivorous) as trophic level
increases. Using our diet specificity index, we
calculated the distance of 
each consumer from equal consumption of brown and green resources as:
\emph{Distance from equally-derived diet} \(= |Diet Index-0.5|\). Thus taxa
that derive their energy equally from both channels would have
a diet index of 0.5 and their distance from an equally-derived diet
(diet-distance) would be 0. This metric allowed us to test for
patterns in diet specialization
with a mixed-effects ANOVA model that included a linear effect of
trophic level, a fixed categorical variable for system type, 
and a random effect for food web identity (accounting for the non-independence of multiple taxa within each web). Because we had an
incomplete set of systems within each
web type (for flow data we had no terrestrial webs, and for binary data
we had no emergent vegetation webs), we did not include web type
(binary/flow) in the analyses. We explored alternative
variance-covariance structures and error distributions
\citep{Bolker:2009, Wolfinger:1996} 
and selected a Gaussian distribution with
autoregressive heterogeneous variances based in
Akaike's Information Criterion (AIC) and examination
of residuals. 
\\ 
\indent We used R version 2.12.0 for all
analyses \citep{Rcore:2010}, and report all summary statistics as mean ${\pm}$
standard error unless otherwise noted.
\\
\\
\emph{Results}\\
\\
\indent We found that multi-channel consumers were prevalent across all ecosystems
(Figure 1), comprising 42.1 ${\pm}$ 5.3\% of taxa (53.5 ${\pm}$ 6.4\%
of all consumers) and were far more
prevalent than taxa feeding only on living
autotrophs (16.8 ${\pm}$ 3.6\%) or detritus (17.5 ${\pm}$ 4.4\%). These
trends were consistent across systems (one-way ANOVA:
F_{3, 19}=1.42, p \(=\) 0.27).
\\
\indent Multi-channel consumers were more common at
higher trophic levels (Figure 2, mixed-effects model, a linear effect
of trophic level:
F_{1, 1130 }= 45.0, p \(<\) 0.0001). This relationship did not vary
by system type (F_{3, 1130 }= 1.17, p \(=\) 0.32), and system type
alone did not explain diet (F_{3, 19}=
0.77, p \(=\) 0.53). As predicted by \citet{Rooney:2006}, multi-channel consumers
were the most common consumer type at the highest trophic levels: 63.6\% of taxa
in trophic level 3 and above were multi-channel consumers consuming at
least 10\% of each resource, as compared to 41.3\% of taxa between the
second and third trophic level. However, when considering where the
majority of multi-channel consumers occurred, we found that most
occupied the first to second heterotrophic level: 58.6 ${\pm}$
5.7\% of all multi-channel consumers occupied trophic levels 2-3, and
nearly three-quarters of these taxa (74.7\%), occurred between trophic
level 2 and 2.5.
This suggests that most multi-channel consumers directly
link basal detrital and autotroph channels. 
\\
\\
\noindent {\bf Effect of multi-channel feeding on system stability}\\
\\
\emph{Methods}\\
\\
\indent Since nearly 60\% of all multi-channel consumers in our
empirical food webs occupied
trophic levels 2-3, we next evaluated how multi-channel feeding low in the food
web affects system stability. We developed a four-compartment
nutrient-recycling predator-prey model (Figure 3) that captured the three features we highlighted in the introduction---a compartmentalized multi-channel structure, primary producers and detritus as basal resources, and consumers (X)
that have the ability to derive energy from both autotrophs (A) and detritus (D). We then linked these compartments to a
plant-available nutrient pool (N). We used a nutrient recycling model because we wanted to evaluate the unique ability of multi-channel feeding to directly bring detrital-web nutrients into the grazing web, which would otherwise occur only via nutrient recycling from the detritus pool. 
\\
\indent In all model formulations autotrophs take up nutrients from a
plant-available nutrient pool. The nutrient pool increases via external
inputs and mineralization from the detrital pool. The detrital pool
increases due to external inputs, sloppy feeding by herbivores and death
from the autotroph and consumer pools. In the multi-channel feeding form of
the model, the consumer feeds on both the autotroph and detrital pool with a Type II functional response
following \cite{Chesson:1983}:

\\
\begin{small}
\begin{align*}
 \frac{dA}{dt} &=\mu NA- \frac{a_{AX}AX}{1+a_{AX}h_{AX}A+a_{DX}h_{DX}D} - d_{A}A- e_{A}A  
\\[2em]
 \frac{dN}{dt} &= I_{N} +
 \frac{(1-\delta_{AX})a_{AX}\gamma_{AX}AX}{1+a_{AX}h_{AX}A+a_{DX}h_{DX}D}
 +
 \frac{(1-\delta_{DX})\gamma_{DX}a_{DX}DX}{1+a_{AX}h_{AX}A+a_{DX}h_{DX}D}+
 mD -\mu NA- e_{N}N
\\[2em]
 \frac{dD}{dt} &= I_{D} +
 \frac{(1-\gamma_{AX})a_{AX}AX}{1+a_{AX}h_{AX}A+a_{DX}h_{DX}D} -
 \frac{\gamma_{DX}a_{DX}DX}{1+a_{AX}h_{AX}A+a_{DX}h_{DX}D} - mD + d_{A}A +
 d_{X}X - e_{D}D
\\[2em]
 \frac{dX}{dt} &=
 \frac{a_{AX}\gamma_{AX}\delta_{AX}AX}{1+a_{AX}h_{AX}A+a_{DX}h_{DX}D} +
 \frac{a_{DX}\gamma_{DX}\delta_{DX}DX}{1+a_{AX}h_{AX}A+a_{DX}h_{DX}D}- d_{X}X - e_{X}X
\end{align*}
\end{small}
% If you want eqns numbered try:
% \begin{equation}
% \frac{dA}{dt} = mNA - a_{AX}AX - dA_{A}A- e_{A}A  
% \\end{equation}

\noindent The model simplifies to pure detritivory when the attack rate
of the consumer on autotrophs (\(a_{AX}\)) is
set to zero, and to pure herbivory when the attack rate
of the consumer on detritus (\(a_{DX}\)) is
set to zero. We used a Type II functional response for realism and because Type I functional responses did not allow analytical solutions (due to the nutrient recycling and omnivory aspects of the model).

\indent We defined parameters for the model (Table 1) for two of our
four system types, freshwater and
terrestrial. These systems are distinctly different endpoints along a
continuum, varying in key attributes that may be important to how
multi-channel feeding affects web stability. Specifically, freshwater systems
tend to have smaller standing stocks of all pools, higher quality
detritus, and faster nutrient cycling as compared to terrestrial
systems \citep{Cebrian:2004}.
\\ 
\indent We designed the equations to be in
expressed in units of the most limiting nutrient (here, \(g\,N\,m^{-2} \,y^{-1}\)
for a terrestrial system and \(\mu g\,P\, L^{-1}\,d^{-1}\) for a freshwater system). We used literature values from grassland
systems and North American temperate lakes and background knowledge to
develop possible ranges for each parameter (Table 1), then explored
parameter space around
these ranges. While some studies have found higher quality detritus in aquatic systems results in higher assimilation and production efficiencies compared to terrestrial systems \citep{Cebrian:2004,Cebrian:2009hg}, our values do not reveal such a pattern. 

\\
\indent We ran separate sets of 100,000 simulations to randomly explore both
terrestrial and freshwater parameter space for the three models: multi-channel feeding (consumption of both detritus and autotrophs by the consumer), detritivory-only, and herbivory only. For each simulation we generated random
parameter sets within uniform distributions between our
minimum and maximum values (Table 1), allowing an examination of a
large parameter space. For each parameter set, we calculated the equilibrium
and assessed whether it was feasible \citep[i.e., all pools had positive
equilibrium values, \emph{sensu}][]{roberts1974}. For systems
with feasible equilibria, we calculated dominant eigenvalues and then used
them to estimate stability (whether the system will return to the
equilibrium if disturbed) and resilience (the rate of recovery)
following classical procedures \citep{May:1973}. While the ecological literature is rife
with definitions and calculations of stability \citep{Grimm:1997},
we chose to use classical procedures
because the dominant eigenvalue has both a clear theoretical definition (that is easily
measurable in the model) and a relationship to empirical
measurements of system's response to perturbation
\citep{Cottingham:1994,Jorgensen:2000}.
\\
\indent We further explored the effect of multi-channel feeding on food web stability by
examining in more detail how system return times
changed with attack rates on the autotroph and detrital pools. We
focused on attack rates because they allowed us to vary how strongly
the omnivore fed on one resource or the other. For
this we used one parameter set for each system type (Table 1), chosen because it
was realistic biologically; results were robust to the choice
of external input rates for both the nutrient and detrital pools. All
model simulations were done in Mathematica 7.0 (Wolfram Research, Inc.) and
analyzed in R 2.12.0.
%Lizzie: edit above as needed.
\\
\\
\emph{Results}\\
\\
\indent Our models suggest that the degree to which multi-channel feeding affects system stability and return time depends on the system type
(terrestrial or freshwater, as reflected by the parameter sets) and the rates at which omnivores attack living
autotrophs versus detritus. In both system types, models with multi-channel feeding produced fewer parameter sets generating stable models compared to models herbivory-only and detritivory-only models (Table 2). 
\newline
\indent The effect of multi-channel feeding on system resilience varied by system type. In terrestrial systems, multi-channel feeding tended to produce systems with
intermediate resilience between the least resilient detritivory-only
models and the most-resilient herbivory-only models. Additionally, in terrestrial
systems multi-channel feeding produced only a small
de-stabilizing effect. For freshwater systems, however, the
destabilizing effect was far greater (Table 2). In freshwater parameter sets with equilibria, 
multi-channel feeding generally produced less resilient systems, compared to models without multi-channel feeding. Return times in models with multi-channel feeding were twice as long as herbivory-only models (which tended to produce systems with the shortest return times) and 50\% higher than detritivory-only models (Table 2). 
\newline
\indent Across both terrestrial and freshwater parameter space, multi-channel feeding was
stabilizing at low attack rates on the autotroph, and destabilizing at high attack rates on the autotroph (Figure 4a-d). When the omnivore
attacked autotrophs at a high rate, the attack rate on
detritus had to be comparatively much lower to produce a stable system
(Figure 4a-d). In models with multi-channel feeding---across both system types---the transition from a stable to unstable system with higher attack rates resulted from system dynamics entering limit cycles, not from the extinction of any pool. 
\\
\\
\noindent {\bf Discussion}\\
\\
\indent Across 23 food webs and four ecosystem types, we found that most
consumers were multi-channel consumers---deriving their diets from both
autotrophs and detritus, especially at the top of the food
web. The majority of multi-channel consumers, however, occurred as primary
consumers, indicating most taxa at the herbivore/detritivore level
are more aptly described as multi-channel consumers. Moreover, our models show that
multi-channel feeding---though generally destabilizing when simplistically compared to herbivory- or detritivory-only models---can be stabilizing, when occurring with low attack
rates on autotrophs (Figure 4). 
\\
\\
\noindent \emph{Prevalence and position of multi-channel consumers in empirical webs}\\
\\
\indent 
Past research has routinely suggested that multi-channel feeding and detrital
resources may be both more common and more important in
terrestrial compared to aquatic ecosystems \citep{Polis:1996}.  In
contrast, we found that multi-channel feeding is as common in
freshwater lakes and oceanic shelves dominated by pelagic species as
in terrestrial systems. Consumers in all four system types derived substantial amounts of energy from both autotroph
and detrital channels. This challenges the traditional view that food
webs can be abstracted into simple grazing channels of plants,
herbivores and predators \citep{Holt:2006,  Pimm:1982}, and
suggests that the real world is far messier, echoing recent work on 
intraguild predation \citep{Rudolf:2007,miller2011}. While
our 23 food webs are still a small sample, our results
suggest that current webs capture highly connected systems. Grazing
chains exist only as one part of webs heavily subsidized by widespread
consumer interactions with the detrital web \citep{Moore:2004}. Thus, while our modeling work showed detritivory-only or herbivory-only systems may be the most stable (Table 2), our empirical web findings suggest that such systems are rare, and are not representative of real systems.
\\
\\
\noindent \emph{Stability \& multi-channel feeding in modeled webs}\\
\\
\indent Our modeling results show that integration between the brown and
green worlds may stabilize food webs under certain conditions. 
Using a simple food web model with nutrient cycling and considering stability as assessed by return times calculated from the
dominant eigenvalue, we found models with multi-channel feeding often produced
stable webs. Such models, however, produced the fewest stable systems (\(<50\%\)); in contrast, herbivory-only systems were consistently more stable (\(54.5\%\) of cases), and all of the detritivory-only models were stable (Table 2). While return times were higher for multi-channel consumer models compared to herbivory-only models, we found that multi-channel feeding could have a stabilizing effect in both terrestrial and freshwater parameter sets, even without the stabilizing forces of
predator-switching often used in other models that introduce this sort of omnivory
\citep{Rooney:2006}. In particular, we found high attack rates on
both autotrophs and detritus led to highly unstable systems, however, multi-channel feeding was stabilizing at low to moderate attack rates on autotrophs.(Figure 4).
\\
\indent We noted, however, distinct differences in the effects of multi-channel feeding on
the stability of terrestrial versus freshwater systems, suggesting that
multi-channel feeding may be generally more stabilizing in terrestrial
systems. Terrestrial systems with multi-channel feeding had return times of intermediate length compared to detritivory and herbivory-only models (Table 2) and showed a peak in stabilizing effects when multi-channel feeders attacked detritus at a low rate (Figure 4b, 4d). In contrast, modeled freshwater systems with multi-channel feeding produced longer return times and showed a larger range of parameter space in which detrital feeding by multi-channel feeding destabilized systems (Figure 4a, 4c). While this may initially seems incongruous with our finding that
multi-channel consumers are equally common across all ecosystem types,
our modeling results suggest that the 
key difference lies in attack rates between systems. Both systems
can be stable with multi-channel consumers and high attack rates on
detritus, provided attack rates on
autotrophs remain low.
\\
\\
\noindent \emph{Integrating results from empirical and modeled food webs}\\
\\
\indent Combining our model
predictions with empirical food web data indicates that while
multi-channel consumers in real webs tend to consume a highly mixed
diet (Figures 1-2), the key for system stability lies in asymmetric
attack rates. Thus, our results integrate the findings that (1) multi-channel feeding can be
stabilizing when weak \citep{McCann:1998} and (2) prey preferences of predators that link food chains affect stability \citep{Post:2000xx}, but also that (3) stable systems have omnivores
which balance their resource needs across dichotomous resource
channels \citep{Rooney:2006}. Dichotomous resource channels can be
critical to stability by allowing multiple pathways and rates of
energy flow through webs. Variation in how dichotomous 
the relative rates of these two channels are may explain differences between
our terrestrial and freshwater parameterized models (Figure 4). Freshwater systems
tend to have higher quality (C:N or C:P) living autotrophs, with
resulting stronger herbivory and faster turnover
times compared to terrestrial systems, while detritus in freshwater
systems is often allochthonous, derived from the lower
quality plant materials of terrestrial systems
\citep{Cebrian:2004}. Thus, the high ratio of edibility of
autotrophs versus detritus in freshwater
systems may produce greater asymmetry in the attack rates of
omnivores: across systems this relative ratio of edibility between the brown
and green webs may be key to predicting the relative asymmetry of attack rates.
\\
\indent Our prediction of trade-offs between attack rates on autotrophs versus
detritus calls for improved data to more 
carefully estimate interactions between
consumers and detrital resources, especially nutrient transfers. Testing our model predictions requires field estimates of
attack rates---especially on detrital resources; particularly
insightful may be data from 
open-water systems where autotrophs, consumers and detritus are all
mobile, and from systems that vary in the quality of their green versus brown
basal resources. Further, improved
estimates of the pool sizes of detritus may be required to calculate accurate attack rates; many webs estimate only a single pool of
detritus, while consumers may view and attack detrital pools of varying quality quite differently \citep{Wilson:2011qo}.
\\
\newline
\noindent \emph{Conclusions:} Our results echo continued work demonstrating the importance
of detritus to structuring food webs
\citep{Allesina:2009,Odum:1984}. Further, while community ecology has generally
conceptualized grazing and detrital webs as separate
\citep{Moore:2004}, our results, combined with increasing empirical
and theoretical work \citep{Anderson:2008, Vadeboncoeur:2005,
 Moore:1988,Blanchard:2011}, suggest consumers across ecosystems ignore this
distinction, drawing resources from both the brown and green
worlds. Our
findings demonstrate that key differences among ecosystems in the effects of multi-channel feeding on stability
and the rates of attack on autotrophs may affect trophic structure. Such differences could affect the flow of nutrients in food webs and webs' dynamical structure, with cascading community  and ecosystem consequences.\\

% We found support for our model prediction--in all webs considered attack rates on autotrophs varied highly but attack rates on detritus were generally low in comparison (Figure 4).

% While multi-channel omnivores appear common in many food webs (Odum and Biever 1984; Pimm 1982) community ecology has generally conceptualized grazing and detrital webs as separate (Moore et al. 2004) and focused most research on grazing webs, although research has continually highlighted the critical role of detritus in food webs (Allesina and Pascual 2009; DeAngelis 1992; Polis and Hurd 1996). This is especially surprising given that detritus-free grazing webs are generally unfeasible because detrital pathways recycle system nutrients (Heal and MacLean 1975) while there are many webs based solely on detritus (Burkepile et al. 2006; Polis and Hurd 1996; Wallace et al. 1997). Even in webs with robust grazing webs, most net productivity is usually shunted to the detrital web (Odum and Biever 1984; Polis and Hurd 1996) further suggesting the role of detrital pathways in food webs must be key (Post 2002).

% Together our findings indicate that connections between grazing and detrital-based webs are a common and critical component to food webs.
\\
\newpage
\noindent {\bf Acknowledgments}\\
\noindent Comments from two anonymous reviewers and S. Diehl greatly
improved the manuscript, for which we are thankful. We also thank J. Dunne and M. Scotti for sharing
food web data and M. O'Connor for reviewing an earlier version of this
manuscript. This work was conducted in part with the Trophic
Structure Comparisons Working Group supported by the National Center
for Ecological Analysis and Synthesis, a Center funded by NSF (Grant
DEB-0072909), the University of California at Santa Barbara, and the
state of California, in part conducted through the The Centre for Biodiversity Theory and Modelling which is supported by the TULIP Laboratory of Excellence (ANR-10-LABX-41), and in part conducted while EMW was an EPA STAR
Fellow and an NSF Postdoctoral Research Fellow in Biology (Grant DBI-0905806), and also while she was supported by the NSERC CREATE training program in biodiversity research. Support for CDM came from 


\newpage
\bibliography{/Users/Lizzie/Documents/EndnoteRelated/Bibtex/LizzieMainMinimal}
\bibliographystyle{/Users/Lizzie/Documents/EndnoteRelated/Bibtex/styles/amnat}
\newpage
% \newgeometry{a4paper, landscape}
%\renewcommand{\familydefault}{\sfdefault}
\noindent \textbf{Table 1:} Parameter values used for four-compartment
nutrient cycling model. An extended version of this table including literature values
and references for all parameters is given in the supplementary
material (Appendix C).
\begin{center}
{\footnotesize 
  \begin{tabular}{ | p{1.5cm} |  p{4cm} | p{4.7cm} || p{4.7cm} | }   
\hline \hline
& & Terrestrial Parameters & Freshwater Parameters \\ \hline
\end{tabular}

  \begin{tabular}{ | p{1.5cm} |  p{4cm} | p{1.75cm} | p{2.5cm} || p{2.25cm} | p{2cm} | }  
\hline 
Parameter & Description & Units  & Values evaluated* & Units & Values evaluated* \\ \hline \hline

\(I_{N}\) & inputs to nutrient pool & \(gNm^{-2}y^{-1}\) & 0.05-10 (0.5) & \(\mu gPL^{-1}d^{-1}\) & 0.001-10 (0.5)\\ \hline

\(I_{D}\) & inputs to detrital pool & \(gNm^{-2}y^{-1}\) & 0.5-20 (1.5)& \(ugPL^{-1}d^{-1}\)  & 0.00005-5 (0.01) \\\hline

\(e_{N}\) &loss rate of inorganic nutrient&\(y^{-1}\) & 	0.005-1.5 (0.01) &	\(d^{-1}\)	&  0.00001-1 (0.05)\\\hline

\(e_{A}\) & loss rate of autotrophs & \(y^{-1}\)&  0.005-1.5 (0.05) & \(d^{-1}\) & 0.00001-1 (0.05)\\\hline

\(e_{D}\) & loss rate of detritus & \(y^{-1}\) & 0.005-1.5 (0.01) & \(d^{-1}\)  & 0.00001-1 (0.05)\\\hline

\(e_{X}\) & loss rate of consumers & \(y^{-1}\)  & 0.001-1.5 (0.1) & \(d^{-1}\) & 0.00001-1 (0.05)\\\hline

\(\mu\) & uptake rate of nutrients by plants & \(m^2y^{-1}g^{-1}\) & 0.5-10 (3) & \(d^{-1}\)  & 0.0001-5 (0.5)\\\hline

\(d_{A}\) & death + metabolic rate of autotrophs & \(y^{-1}\)  & 0.001-4 (0.02) & \(d^{-1}\)  & 0.0001-1 (0.01)\\\hline

\(d_{X}\) & death + metabolic rate of consumers & \(y^{-1}\) & 0.001-5 (0.01) & \(d^{-1}\) & 0.0001-1 (0.05)\\\hline

\(\gamma_{AX}\) & assimilation efficiency  feeding on autotrophs & unitless & 0.2-0.9 (0.3) & unitless   & 0.1-0.9 (0.5)\\\hline

\(\gamma_{DX}\) & assimilation efficiency  feeding on detritus & unitless & 0.2-0.9 (0.5) & unitless &  0.1-0.9 (0.5)\\\hline

\(\delta_{AX}\) & production efficiency  feeding on autotrophs & unitless & 0.3-0.7 (0.35) & unitless   & 0.3-0.7 (0.4)\\\hline

\(\delta_{DX}\) & production efficiency feeding on detritus & unitless &  0.3-0.7 (0.5) & unitless &  0.2-0.9 (0.4)\\\hline

\(a_{AX}\) & attack rate  on autotrophs & \(m^{2}y^{-1}g^{-1}\) & 0.05-5 (1.5) & \(\mu gP^{-1}L^{-1}d^{-1}\) &  0.0001-6 (0.1)\\\hline

\(a_{DX}\) & attack rate on detritus & \(m^{2}y^{-1}g^{-1}\) & 0.05-5 (varied) & \(\mu gP^{-1}L^{-1}d^{-1}\) & 0.0001-6 (varied)\\\hline

\(m\) & detritus mineralization rate & \(y^{-1}\) & 0.3-10 (1.5) & \(d^{-1}\) &  0.001-5 (0.05)\\
  \hline
\(h_{AX}\) & handling time on autotrophs & \(y\) & 0.0000001-1 (0.1) &
\(d\) & 0.01-100 (0.1)\\
  \hline
\(h_{DX}\) & handling time on detritus & \(y\) & 0.0000001-1 (0.1) &
\(d\) &  0.01-200 (0.1)\\
  \hline
  \end{tabular}}
\end{center}

\noindent {\footnotesize * Values evaluated give the range examined for most results, then parenthetically the value used to test sensitivity to attack rates.}

\newpage
\noindent \textbf{Table 2:} Parameter search results: we give the percent of systems with negative critical eigenvalues (\% stable systems) and their mean return time for freshwater and terrestrial parameter sets.
\begin{center}
{\footnotesize 
  \begin{tabular}{| p{4.5cm} | p{2cm}  p{2cm} || p{2cm} p{2cm} |  }   
\hline \hline
 & Freshwater & & Terrestrial  &\\ \hline
\end{tabular}
  \begin{tabular}{ | p{4.5cm} |  p{2cm} | p{2cm} || p{2cm} | p{2cm} | }   
\hline \hline
 & \% stable systems & mean return time & \% stable systems & mean return time \\ \hline
Detritivory-only model & 100\% & 5.0 & 100\% & 2.6 \\ \hline
Herbivory-only model & 54.5\% & 3.5 & 80.8\% & 1.9\\ \hline
Multi-channel feeding model & 45.8\% & 7.5 & 71.6\% & 2.4 \\ \hline
\end{tabular}}
\end{center}

% \restoregeometry

\newpage
\noindent {\bf Figure Legends}\\

\\
\noindent {\bf Figure 1}: All 23 food webs we examined were based on
both detritus (trophic level=1 and diet=0) and living autotrophs
(trophic level=1 and diet=1). Contrary to expectations, there were
many taxa at the second trophic level that mixed their diet between
detrital and living autotroph derived resources. \\
\\
\noindent {\bf Figure 2}: Data from 23 food webs support the
hypothesis that higher trophic levels have less specialized
(more omnivorous) diets than lower trophic levels. Trophic level, on
the x-axis, starts at 2 (since trophic level 1 is constrained to be
detritus or living autotrophs) and diet-distance is on the y axis
(where 0 is eating 50\% from the brown web and 50\% from the green web, and 0.5 is
eating completely from either the detrital or autotroph-based
resources). Lines represent fits from the mixed-effect model presented
in the text.\\
\\
\noindent {\bf Figure 3}: To examine how multi-channel feeding may
affect food web stability we used a four-compartment nutrient recycling
model that varied whether or not the consumer ate detritus (thick dashed
arrow). Thick black arrows represent internal flows, while thick gray arrows
represent inputs and outputs; thinner arrows show flows
to the detrital pool via death (solid gray lines) and sloppy feeding (solid black lines) and to the nutrient
pool via excretion (dashed lines). \\

\noindent {\bf Figure 4}: Contour plots of return times
 depending on attack rates on autotrophs 
versus attack rates on detritus by a multi-channel consumer within freshwater (a and c)
and terrestrial (b and d) parameter sets (parameter values given in Table 1). 
Black contour lines show return times; the shaded gray areas represent parameter space where systems were unstable while unshaded areas represent parameter space where systems were
stable. Overlay areas with diagonal red lines represents parameter space where
multi-channel feeding is stabilizing (versus destabilizing) compared to a herbivory-only (a-b) or a detritivory-only model (c-d).\\

\newpage
\noindent {\bf Figure 1}
\\
\begin{figure}[h!]
\centering
\noindent \includegraphics[width=1.1\textwidth]{/Users/Lizzie/Documents/Professional/NCEAS/Manuscripts/Brown_Green_Omnivory/Figures/Fig1_BG1_Systems.png}
\end{figure}

\newpage
\noindent {\bf Figure 2}
\\
\begin{figure}[h!]
\centering
\noindent \includegraphics[width=.9\textwidth]{/Users/Lizzie/Documents/Professional/NCEAS/Manuscripts/Brown_Green_Omnivory/Figures/Fig2_BG1_5050.png}
\end{figure}

\newpage
\noindent {\bf Figure 3}
\\
\begin{figure}[h!]
\centering
\noindent \includegraphics[width=.5\textwidth]{/Users/Lizzie/Documents/Professional/NCEAS/Manuscripts/Brown_Green_Omnivory/Figures/ModelDiagram/BrGr_ModDiagram_Full.png}
\end{figure}

\newpage
\noindent {\bf Figure 4}
\\
\begin{figure}[h!]
\centering
\noindent \includegraphics[width=1\textwidth]{/Users/Lizzie/Documents/Professional/NCEAS/Manuscripts/Brown_Green_Omnivory/Figures/Fig3_contours_combinedwshading.png}
\end{figure}


\end{document}
